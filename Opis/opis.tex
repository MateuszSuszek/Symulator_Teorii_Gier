\documentclass[a4paper]{article}
% Kodowanie latain 2
%\usepackage[latin2]{inputenc}
\usepackage[T1]{fontenc}
% Można też użyć UTF-8
\usepackage[utf8]{inputenc}

% Język
\usepackage[english]{babel}

\usepackage{graphics}
% - szersza strona
\usepackage[nofoot,hdivide={2cm,*,2cm},vdivide={2cm,*,2cm}]{geometry}

\usepackage{hyperref}

\pagestyle{empty}

% dane autora
\author{Mateusz Suszek}
\title{Wstęp do programowania w języku C \\ Opis projektu}
\date{}

\begin{document}
	
\Large{
	
	\maketitle
	
	\thispagestyle{empty}
	
	W ramach projektu stworzę symulator gier z kombinatorycznej teorii gier przy użyciu GTK oraz Glade.
	Program umożliwi użytkownikowi grę w wybrane gry przeciwko komputerowi lub innemu graczowi. \\
	
	Dla każdej gry gracz będzie miał możliwość:
	\begin{enumerate}
		\item[$\bullet$] Dostosować rozmiar planszy,
		\item[$\bullet$] Zdecydować o tym, kto rozpocznie grę,
		\item[$\bullet$] Wybrać, czy przeciwnikiem steruje komputer,
		\item[$\bullet$] Zapisać oraz wczytać stan gry, który zostanie zachowany pomiędzy sesjami.\\
	\end{enumerate}
	 
	Obsługiwane gry:
	
	\begin{enumerate}
		\item[$\bullet$] Nim oraz jego warianty,
		\item[$\bullet$] Uproszczona wersja gry Obstruction\footnote{\url{http://www.lkozma.net/game.html}} $(1\times n)$,
		\item[$\bullet$] Domineerig.
	\end{enumerate}
	Postaram się w miarę możliwości poszerzyć tę listę.\\
	
	Obliczanie strategii wygrywającej oraz reprezentacja graficzna każdej z tych gier jest podobna, zatem podzielenie kodu na odpowiednie moduły odpowiedzialne za poszczególne zadania umożliwi znacznie szybsze dodawanie nowych gier do programu, wymagające jedynie niewielkich modyfikacji.
	
	

}
\end{document}
